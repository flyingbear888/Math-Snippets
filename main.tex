\documentclass[a5paper]{article}
%\documentclass[13pt, a4paper]{extarticle}

\usepackage[margin=0.25in]{geometry}

\usepackage[english]{babel}
\usepackage[utf8x]{inputenc}
\usepackage{amsmath}
\usepackage{graphicx}
\usepackage[colorinlistoftodos]{todonotes}

\usepackage{ulem}

\usepackage{indentfirst}


\usepackage{./mystyle}

\title{Math Snippets}
\author{Wei Sun}

\begin{document}

%\huge

\maketitle

%\begin{abstract}
%Your abstract.
%\end{abstract}

\section{Introduction}


\section{Topological $K$-theory, algebraic $K$-theory, Bott periodicity and the six-term exact sequence}


\subsection{Topological $K$-theory}


Topological $K$-theory is for compact Hausdorff spaces.

\twomm

Let $X$ be a compact Hausdorff spaces. We define $K^*(X)$ to be made up of the Gronthendick group of $V(X)$, where $V(X)$, roughly speaking, is the semi-group formed by the complex vector bundles over $X$.

Note that we use $K^*(X)$ instead of $K_*(X)$ because this is a contravariant functor.


The reason we want $X$ to be compact is \todo{QQQ} ???

The reason we want $X$ to be Hausdorff is \todo{QQQ} ???

The addition on $V(X)$ is defined in the natural way. Just need to note that this addition might not satisfy cancellation, which is a deep topological fact.

About Bott periodicity.

Let $X$ be a compact Hausdorff space. We have already defined $K(X)$, which is an abelian group (For more details, we start with an abelian semigroup, then Gronthedick it).

Besides, as in the Atiyah book, a ring structure is also defined for $K(X)$. Roughly speaking, given two complex vector bundes $E$ and $F$ over $X$, we can define the vector bundle $E \otimes F$, which is a complex vector bundle over $X$.






%\begin{figure}
%\centering
%\includegraphics[width=0.3\textwidth]{frog.jpg}
%\caption{\label{fig:frog}This frog was uploaded via the project menu.}
%\end{figure}


%\subsection{How to add Comments}

%Comments can be added to your project by clicking on the comment icon in the toolbar above. 

% * <john.hammersley@gmail.com> 2016-07-03T09:54:16.211Z:
%
% Here's an example comment!
%
%To reply to a comment, simply click the reply button in the lower right corner of the comment, and you can close them when you're done.

%Comments can also be added to the margins of the compiled PDF using the todo command, \todo{Here's a comment in the margin!} as shown in the example on the right. You can also add inline comments:

%\todo[inline, color=green!40]{This is an inline comment.}




\section{On topological entropy}




\subsection{On definitions and basics topological entropies}

0. Below is some basics on basics, which is related to the definition of topological entropy.

\twomm

Some basic facts we have:

\vspace{2mm}

{\bf Fact:} If there is a sequence $\{ a_n \}$ such that $a_{m + n} \leq a_m + a_n + C$ for certain constant $C$ and for all $m, n$. Then $\lim \frac{a_n}{n}$ always exists and this limit can never be $+ \infty$. 

\vspace{2mm}

Note: For a sequence satisfying $a_{m + n} \leq a_m + a_n + C$ for all $m, n$, one example is just a decreasing positive sequence.

\vspace{2mm}


{\bf Note}: Under the same setup, it is possible that $\lim \frac{a_n}{n}$ might be $- \infty$. For example, consider $a_n = - n^2$ and with $C = 0$.


\vspace{2mm}


1. Topological entropy is invariant under conjugacies (as the value is defined as the supremum of blabla).

\twomm

2. \sout{The topological entropy of $\alpha$ and $\alpha^{-1}$ might not be the same (I should have a concrete example for it).}

	Using $H(X, \alpha)$ to denote the topological entropy of dynamical system $(X, \alpha)$, then $H(X, \alpha) = H(X, \alpha^{-1})$.
    
    That is because of the following fact:
    
    $$ \#(C \vee \alpha^{-1}(C) \vee \cdots \vee \alpha^{k}(C) = \#(\alpha^{k}(C) \vee \alpha^{k - 1}(C) \vee \cdots \vee C) $$
    
    Note: As for mean dimension, considering a homeomorphism $T$ over the space $X$, then the mean dimension of $T$ equals the mean dimension of $T^{-n}$.

\twomm

3. Given a compact Hausdorff space $X$, consider the range of topological entropy, is it surely an interval??
	
    Definitely not. Consider the space made up of finitely many point with discrete topology on it.
    
	Fact: For certain space $X$, we know the range of entropies for all homeomorphisms is not an interval.
    
    For example, let $X = \T$. Then each irrational rotation has entropy zero.


\twomm

4. Given a compact Hausdorff space $X$, the topological entropy function (over homeomorphisms) might not be continuous as for homeomorphisms on $X$. One typical example is the rotations on $\T$.

\twomm

5. How to derive

\twomm


6. As indicated in the big textbook, the topological entropy of all homeomorphisms on the circle $\T$ is zero. How to prove it?

	
   In case it is a rational rotation, easy to show that the topological entropy is zero. Just check the definition.
   
   What if it is an irrational rotation? What is the entropy of irrational rotation? %How the mixing is done? Do we have any ``patterns"?
   
   One cheating approach is to use the ``metric version'' of definition for topological entropy. For irrational rotations, as it preserve the metric, it follows that they all have topological entropy zero.
  
What to do in this case?


    Find something that works like a charm?
    
    What can we do?
    
  	
    
\section{KK-theory}

KK-theory. It is bi-variant homology/cohomology theory.

QQQ....  yyy

Before dealing with KK-theory, it might be a good idea to deal with K-theory first.

We start with topological $K$-theory for compact Hausdorff space $X$. 

... QQQ??? zzz

So far, we have defined and discussed $K( C(X) )$ (indeed, we covered $K^*(X)$, which is not a strict cohomology theory as it violates the Dimension Axiom.

Given something of ...






    
    
   
   

   
   
   











\end{document}